\section{Integralrechnung}

\subsection{Riemann Integrierbarkeit}
Das Riemann Integral zerlegt die Funktion $f$ in \textit{unendlich} viele Rechtecke mit Breite \textit{0} und einer zufälligen Höhe $\xi_i$. Die Summe aller Rechtecke konvergiert gegen die Fläche unter der Funktionslinie (Das Integral $I$).
\[
\int\limits_{a \text{ \tiny(untere Grenze)}}^{b \text{ \tiny(obere Grenze)}}f(x)dx = \lim_{\substack{n\to\infty\\\Delta x_i\to 0}} \sum\limits_{i = 1}^{n}f(\xi_i)\underbrace{(x_i - x_{i-1})}_{\Delta x_i}
\]

\subsubsection{Integrierbarkeit}
Eine Funktion $f$ ist integrierbar, wenn sie stetig oder monoton beschränkt und an höchstens endlich vielen Stellen unstetig ist.

\subsection{Bestimmtes Integral}
\[\int_{a}^{b}f(x)dx = F(b) - F(a) = \int_{a}^{0}f(x)dx + \int_{0}^{b}f(x)dx\]

\subsection{Formeln}
\noindent Flächeninhalt: \[A = \int_{a}^{b}\left|f(x)\right|dx\]
\noindent Integral Ableitung mit Funktion-Limits: \[\left(\int_{a(x)}^{b(x)}f(t)dt\right)' = f(b(x))b'(x) - f(a(x))a'(x)\]
\noindent Integral Ungleichung:
\[\left|\int_{a}^{b}f(x)dx\right| \leq \int_{a}^{b}\left|f(x)\right|dx\]

\subsection{Mittelwertsatz}
Sei $f(x)$ in $[a;b]$ stetig, dann ist der Mittelwert aller Punkte:
\[\frac{1}{b-a}\int_{a}^{b}f(x)dx\]

