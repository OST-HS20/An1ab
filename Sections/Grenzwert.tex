\section{Grenzwert}
\subsection{Formen}
\noindent\textbf{Bestimmt}
\begin{align*}
	\frac{g}{\infty} = 0,\; 
	\infty^\infty = \infty,\; 
	\frac{\infty}{0^\pm} = \pm\infty,\;
	\frac{1}{0^\pm} = \pm\infty,\;
	\\
	\frac{g}{0^+} = \begin{cases} \infty & g > 0\\-\infty & g < 0 \end{cases},\;
	\frac{\infty}{g} = \begin{cases} \infty & g > 0\\-\infty & g < 0 \end{cases}
\end{align*}

\noindent\textbf{Unbestimmt}
\begin{align*}
	\frac{0}{0},\;
	\frac{\infty}{\infty},\;
	0\cdot\infty,\;
	\infty - \infty,\;
	0^0,\; \infty^0,\;
	1^\infty
\end{align*}
siehe \verweiseref{lhopital} für die Bestimmung von \textit{unbestimmten} Grenzwerten

\subsection{Einschliessungs-Prinzip}\label{einschliessungsprinzip}
Wenn eine Funktionen $f(x)$ in $a(x) < f(x) < b(x)$ geschlossen ist, und wenn $\lim\limits_{x \rightarrow g}a(x) = G$ sowie $\lim\limits_{x \rightarrow g}b(x) = G$, dann ist auch $\lim\limits_{x \rightarrow g}f(x) = G$. \bronstein{56}

\subsection{Bolzano-Weierstrass}\label{bolzano}
Wenn eine Folge \underline{monoton} (Siehe \verweiseref{monotonie}) \textbf{und} \underline{beschränkt} (Siehe \verweiseref{beschränkt}) ist, dann ist diese auch Konvergent. Hinweis: Grenzwertgleichung alle $z_n$ durch $g$ ersetzten und nach $g$ Auflösen, $g$ muss im Intervall liegen!

\[
\left.\begin{aligned}
	\exists\sup f(x) &\text{ und } f(x) \Uparrow \\
	\exists\inf f(x) &\text{ und } f(x) \Downarrow
\end{aligned}
\right\rbrace \Rightarrow 
f(x) \text{konvergent}
\]

\noindent Supremum: kleinste obere Schranke, Infimum: grösste untere Schranke

\subsection{Bemerkenswerte Grenzwerte}
\begin{align*}
	\setlength\extrarowheight{8pt}
	\begin{array}{*2{>{\displaystyle}l}}
		\lim_{x\to 0} \frac{\sin x}{x} = 1 & \lim_{x\to\infty} \left(1 + \frac{\textcolor{red}{\pm a}}{x}\right)^x = e^{\textcolor{red}{\pm a}\texttt{}} \\
		\lim_{x\to 0} \frac{a^x - 1}{x} = \ln a & \lim_{x\to\infty} \frac{(\ln x)^a}{x^b} = 0 \\
		\lim_{x\to 0} \frac{e^x - 1}{2} = 0 & \lim_{x\to\infty} \sqrt[x]{p} = 1\\
		\lim_{x\to 0} x \cdot \ln x = 0 & \lim_{x\to\infty} \sum_{k=0}^x q^k = \frac{1}{1-q} \quad (|q| < 1)\\
		\lim_{x\to\infty} \sqrt[x]{x} = 1 & \lim_{x\to\infty} \left(\frac{K^x}{x!}\right) = 0 \qquad K\text{ konstant}\\
		\lim_{x\to\infty} \sqrt[x]{x!} = +\infty & \lim_{x\to\infty} \sqrt[x]{\frac{K^x}{x!}} = 0 \qquad K\text{ konstant} > 0\\
		\lim_{x\to\infty} \frac{x}{\sqrt[x]{x!}} = e \\
	\end{array}
\end{align*}

\subsection{Bernoulli-l’Hôpital}\label{lhopital}
Wenn $\frac{f(x)}{g(x)} = \frac{0}{0}$ oder $\frac{\pm\infty}{\pm\infty}$ und $g'(x) \neq 0$ dann gilt:
\[
\lim\limits_{x \rightarrow a}\frac{f(x)}{g(x)} = \lim\limits_{x \rightarrow a}\frac{f'(x)}{g'(x)}
\]
Falls die Ableitung noch unbestimmt ist, einfach Wiederholen. \bronstein{57}

\noindent\textbf{Hinweis}\\
\begin{tabular}{rrl}
	$f \cdot g\rightarrow $ & $ (0+) \cdot \infty$: & $\frac{f}{1 / g}$ vom Type $\frac{0}{0}$; $\frac{1/f}{g}$ vom Type $\frac{\infty}{\infty}$ \\ && \\
	$f - g \rightarrow $& $\infty - \infty$: & $\frac{1/g-1/f}{1/(f\cdot g)}$ vom Type $\frac{0}{0}$ \\ && \\
	$f^z \rightarrow$ & $ (0+)^0, \infty^0, 1^\infty$: & $e^{z\cdot\ln f}$
\end{tabular}\\ \\ 
Durch Ausklammern von Vorzeichen und Reziprokbildung bei negativen Exponenten sind andere Vorzeichenvarianten vollständig abgedeckt.
